\documentclass{article}

\usepackage[english]{babel}

\usepackage[letterpaper,top=2cm,bottom=2cm,left=3cm,right=3cm,marginparwidth=1.75cm]{geometry}

\usepackage{amsmath}
\usepackage{graphicx}
\usepackage{enumitem}
\usepackage[colorlinks=true, allcolors=blue]{hyperref}
\usepackage{natbib}
\bibliographystyle{apalike}
\usepackage{caption}
\usepackage{float}
\usepackage{xcolor}
\usepackage{listings}

\definecolor{mGreen}{rgb}{0,0.6,0}
\definecolor{mGray}{rgb}{0.5,0.5,0.5}
\definecolor{mPurple}{rgb}{0.58,0,0.82}
\definecolor{backgroundColour}{rgb}{0.95,0.95,0.92}

\lstdefinestyle{CStyle}{
    backgroundcolor=\color{backgroundColour},   
    commentstyle=\color{mGreen},
    keywordstyle=\color{magenta},
    numberstyle=\tiny\color{mGray},
    stringstyle=\color{mPurple},
    basicstyle=\footnotesize,
    breakatwhitespace=false,         
    breaklines=true,                 
    captionpos=b,                    
    keepspaces=true,                 
    numbers=left,                    
    numbersep=5pt,                  
    showspaces=false,                
    showstringspaces=false,
    showtabs=false,                  
    tabsize=2,
    language=C
}

\lstset{
    language=Prolog,
    basicstyle=\small\ttfamily,
    keywordstyle=\bfseries,
    commentstyle=\itshape,
    numbers=left,
    numberstyle=\tiny,
    numbersep=5pt,
    frame=tb,
    breaklines=true,
    showstringspaces=false
}

\lstnewenvironment{prologcode}

\title{Linguagens de Programação - Professor Haniel Barbosa \\ Lista de exercícios 4}
\author{Luís Felipe Ramos Ferreira \\ 2019022553}

\begin{document}
\maketitle

\section{Passagem de Parâmetros}
    \begin{enumerate}
        \item
            \begin{enumerate}[label=(\alph*)]
                \item
                    O valor impresso pelo programa é 5
                \item
                    O valor impresso nesse caso é 7. Como \textit{C++} possui escopo estático,
                    o valor da variável x utilizado dentro da função \texttt{p} será o definido globalmente, isto é,
                    onde x é giaul a zero. O variável x dentro do escopo da função \textit{main} terá sua referência
                    passada para os dois parâmetros da função \texttt{p}. Dentro do escopo da função, o valor da vriável global
                    x será incrementada para 1, e as variáveis y e x serão incrementadas. Como elas são referências para a variável x com
                    valor igual a 1 no escopo da função \texttt{main}, esse mesmo valor será incrementado 2 vezes. Podemos entender x e y
                    como duas variáveis que apontam para a mesma posição da memória. Dessa maneira, o valor nessa posição da memória será incrementado
                    para 3, e ele será somado duas vezes no comando de impressão. Dessa forma, teremos uma saída com valor 1 + 3 + 3, ou então 7.
            \end{enumerate}
        \item
            \begin{enumerate}[label=(\alph*)]
                \item
                    \begin{lstlisting}[style=CStyle]
                        #define SUM(X, Y) (X) + (Y)
                        int main(int argc, char** argv){
                        printf("sum = %d\n", {(argc) + (argv[0][0])}); 
                        }
                    \end{lstlisting}
                \item 
                    A captura de variáveis é um problema que ocorre na expansão de macros quando dentro da definição da macro há
                    a definição de uma variável cujo nome está sendo utilizado por outra variável passada como parâmetro para a macro, causando
                    um conflito entre a qual variável cada operação se refere. Um exemplo da situação apresentado em sala foi
                    o do uso da macro \textit{SWAP}, que define uma função para a troca de valores entre duas variáveis, e que dentro do escopo
                    da macro define uma variável auxiliar denominada \texttt{temp}. Abaixo, o código disponibilizado nas notas de aula
                    pode ser visto:
                    
                    \begin{lstlisting}[style=CStyle]
                        #include "stdio.h"
                        #define SWAP(X, Y)   \
                            {                \
                                int tmp = X; \
                                X = Y;       \
                                Y = tmp;     \
                            }
                        int main()
                        {
                            int a = 2;
                            int tmp = 15;
                            printf("Before: %d, %d\n", a, tmp);
                            SWAP(a, tmp);
                            printf("After: %d, %d\n", a, tmp);
                        }
                    \end{lstlisting}
                    
                    Neste cenário, após o pré processamento, o código da maneira a seguir. Nela, podemos ver que
                    há a captura da variável \texttt{tmp}, e o comportamento não é o desejado, uma vez que a variável auxiliar
                    \texttt{tmp} que é definida dentro da macro encobre e faz com que a variável \texttt{tmp} original, passada como
                    parâmetro, se perca, e assim o \textit{swap} de variáveis não ocorre como esperado.
                    
                    \begin{lstlisting}[style=CStyle]
                        #include "stdio.h"
                        #define SWAP(X, Y)   \
                            {                \
                                int tmp = X; \
                                X = Y;       \
                                Y = tmp;     \
                            }
                        int main()
                        {
                            int a = 2;
                            int tmp = 15;
                            printf("Before: %d, %d\n", a, tmp);
                            {
                                int tmp = a;
                                a = tmp;
                                tmp = tmp;
                            }
                            printf("After: %d, %d\n", a, tmp);
                        }
                    \end{lstlisting}
                \item
                    O código em \texit{C} abaixo contêm o problema da múltipla avaliação de parâmetros. Ele foi apresentado e
                    discutido durante as aulas da disciplina.
                    
                    \begin{lstlisting}[style=CStyle]
                        #include "stdio.h"
                        int x = 0;
                        int foo()
                        {
                            x++;
                            return 1;
                        }
                        #define MAX(X, Y) ((X) > (Y) ? (X) : (Y))
                        int main()
                        {
                            int y = MAX(0, foo());
                            printf("Max: %d, global x: %d\n", y, x);
                        }
                    \end{lstlisting}

                    Após a expansão da macro, como os argumentos passados irão ser substituídos no corpo da
                    macro, a função \texttt{foo} será chamada duas vezes, o que irá impactar no incremento da variável 
                    global \texttt{x}. A saída do código a seguir seria então:
                    
                    Max: 1, global x: 2
            \end{enumerate}

            \item
                \begin{enumerate}[label=(\alph*)]
                    \item
                        O valor de \texttt{z} na chamada é 30
                    \item
                        O valor 
                \end{enumerate}
            \item
                \begin{enumerate}[label=(\alph*)]
                    \item
                        m1.i = 4 e m2.i = 4
                    \item
                        m1.i = 3 e m2.i = 3
                    \item
                        m1.i = 4 e m2.i = 4
                    \item
                        Java adota o tipo de passagem por valor para tipos primitivos
                    \item
                        Java adota o tipo de passagem por valor para objetos
                \end{enumerate}
        
    \end{enumerate}

\section{Programação Lógica}

\begin{enumerate}
    \item
        \begin{enumerate}[label=(\alph*)]
            \item
                \begin{lstlisting}[language=Prolog]
                    firstCousin(X, Y) :- sibling(M, N), parent(M, X), parent(N, Y), not(X=Y), not(sibling(X, Y)).
                \end{lstlisting}
            \item
                \begin{lstlisting}[language=Prolog]
                    descendant(X, Y) :- parent(Y, X); (descendant(X, K), parent(Y, K)).
                \end{lstlisting}
        \end{enumerate}
    \item

        \begin{lstlisting}[language=Prolog]
            third([_|[_|[Y|_]]], Y).
        \end{lstlisting}
    \item
        \begin{lstlisting}[language=Prolog]
            dupList([], []).
            dupList([H|T], [G|[G|K]]) :- dupList(T, K), H=G.
        \end{lstlisting}
    \item
        \begin{lstlisting}[language=Prolog]
            third([_|[_|[Y|_]]], Y).
        \end{lstlisting}
\end{enumerate}
    
\end{document}
